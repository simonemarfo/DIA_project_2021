We have done a script file for every step of the project \textit{(n1.py, n3.py, n4.py, n5.py, n6.py, n7.py, n8.py)}. The first two steps are theoretical and they are included in this report, by the way we have implemented, for completeness, a script for the first step that find the optimal solution of the offline optimization problem.\\
In addition, we implemented the following codes, in order to compute the presented results. 
\subparagraph*{Config.py} is the configuration files and contains:
\begin{itemize}
	\item The samples used to reconstruct the demand curves of the two items
	\item The parameters of the customer distribution
	\item Other description information about the environment
\end{itemize} 

\subparagraph*{Context.py}
Context.py contains the definition of the class that simulates the behaviour of the context. All the experiments use this class. It is initialized with the values written in the config.py. It implements the methods related to the purchase of the two items based on their conversion rates, the generation of the daily customer numbers per class, the algorithm to calculate the optimal solutions for the non-stationary environment and the plot functions for the demand curves and the other distributions.


\subparagraph*{Algorithms folder}
The Algorithms folder contains all the learning algorithms: UCB and Thompson Sampling, with their customized version.

\subparagraph*{DIAnotebook.pdf}
It contains all the script files with the relative results.